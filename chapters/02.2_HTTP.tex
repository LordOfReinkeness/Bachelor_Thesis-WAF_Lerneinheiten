% HTTP (Hypertext Transfer Protocol)
% - Übertragung von Web-Inhalten (Daten für websites)
% - Client-Server-Kommunikation: Clients fragen Webinhalte von Servern an
% - Request-Response Pattern
% - Stateless: Request-Response unabhängig, Kommunikation nicht wiederaufnehmbar
%
% Erweiterungen seit Version 2
% - Multiplexing and Stream Prioritization: 
%     - Mehrere Anfragen und Antworten gleichzeitig
%     - TCP Erweiterung
%     - blocking
% - Push Benachrichtigung:
%     -  Server mehrere Responses auf eine Anfrage
%     - Verbindungen werden nicht sofort geschlossen
% - QUIC 
%     - transport layer Protokoll
%     - nicht mehr TCP basiert
% 
% **Für WAF HTTP v1 Relevant**
% - Grundlegendes HTTP schema verstehen
% 
% **Protokoll**
% Kommunikationsprozess:
% 1. Verbindungsaufbau
%     - Client -> Server
% 2. Anfrage 
%     - Client fragt Daten an oder löst Aktion aus
% 3. Antwort 
%     - Server quittiert Anfrage
%     - eventuell Daten als Antwort
% 4. Verbindung schließen
%     - Server kann eigenständig keine Verbindung zu client mehr herstellen
% 
% Kommunikationsschema:
% 
% HTTP-Request
% - Startzeile:
%     - Methode: 
%         - Aktion (GET, POST, ...)
%         - angelehnt an SQL Datenbank Operationen
%     - Anfrage-URL
%         - Pfad entsprechen UNIX-Path
%         - Prameter (filter) nach `?`
%     - HTTP-Version: 
%         - Version der folgenden Kommunikation
% - Header
% - Body
% 
% HTTP-Response
% - Statuszeile:
%     - Status Code: 
%         - Numerischer Code
%         - Status der Verarbeitung (200:OK, 400 User Fehler, 500: Server Fehler, ...)
%     - Status-Text: 
%         - Beschreibung des Status code
% - Header
% - Body
% 
% HTTP-Header:
% - zusätzliche Informationen (Host, User-Agent, Content-Type, ..)
% - Enthalten Metadaten über die Anfrage oder den Client/Server.
% 
% HTTP-Body
% - Optional
% - Daten

Das Hypertext Transfer Protokoll (HTTP) ist ein Protokoll in der Internet-Kommunikation, dass zur übertagung diverser Daten unterschiedlicher Datentypen genutzt werden kann. Sein Haupt-Einsatzgebiet ist die Datenübertragung zwischen Webseiten und Clients.
Seit der Einführung in 1991 wurde es in mehreren RFCs erweitert und ist inzwischen in version drei.\\

In seiner aktuellen Form kann es Gebrauch von TCP-Sitzungen machen um Verbindungen über längere Zeit aufrecht zu halten und fortgeschrittenere Kommunikation, wie push Nachrichten, zu erlauben. Außerdem kann TCP-Pipelining genutzt werden um die parallele Abarbeitung von Anfragen zu ermöglichen und nicht auf die \textit{Acknowledge}-Nachrichten von TCP warten zu müssen.
Um seine Grundfunktion zu erläutern wird in diesem Kapitel die statuslose Kommunikation, definiert in Version 1.0 die mit einem unmittelbaren Anfrage-Antwort Muster arbeitet, beschrieben.
Hierin initiiert ein Client mit einer Anfrage Nachricht \footnote{Im folgenden werden HTTP-Anfrage und das Englische HTTP-request synonym verwendet} eine Verbindung, die von einem Server verarbeitet und mit einer Antwort-Nachricht beantwortet wird. Darauf wird die Verbindung geschlossen.
Dem Sever ist es nun nicht mehr möglich dem Client weitere Daten zu senden, ohne das der Client eine weitere Anfrage-Nachricht schickt.

\paragraph{HTTP-Nachrichten}
Die Grundlegende Einheit einer \ac{http}-Kommunikation wird als \textit{Nachricht} bezeichnet.
Da \ac{http} ein Klartext-Protokoll ist, werden diese in menschenlesbarer Form als Text übertragen.
Eine Nachricht besteht aus einer \textit{Start-Zeile}, die die Nachricht entweder als Anfrage oder Antwort identifiziert. In diesen beiden Fällen hat die Zeile jeweils einen untersiedlichen Aufbau:

\begin{description}
     \item[Request-Zeile:] Ein HTTP-Request ist durch eine \textit{Request-Zeile} identifiziert. Diese ist in drei Teile Aufgeteilt.
     \begin{description}
          \item[Die HTTP-Methode:] beschreibt die   
     \end{description}
     \item[Status-Zeile:] 
\end{description}


\pagebreak    