Sie haben sich im ersten Kapitel mit der Funktion der Laborumgebung vertraut gemacht und erste Regeln verfasst.
Dieses Kapitel befasst sich nun mit den ersten Angriffen die in ähnlicher Form auch an öffentlich erreichbaren Servern beobachtet werden können.
Im Folgenden werden Sie sich \textit{SQL-Injections} und \textit{Cross Site Scriptings (XSS)} genauer anschauen und WAF-Regeln für Angriffsszenarien im \textit{Juice Shop} schreiben, die diese in den Dort aufgeführten Fällen mittigieren.

\subsection{SQL Injections}
Die \href{https://portswigger.net/web-security/sql-injection/cheat-sheet}{SQL-Injection} ist eine sehr häufig auftretende Schwachstelle, die aus der unsauberen Weitergabe von Nutzerdaten an eine SQL Datenbank entstehen kann.

\subsubsection{SQL-Login-Bypass}
In der Challenge \glqq\href{https://pwning.owasp-juice.shop/companion-guide/latest/part2/injection.html#_log_in_with_the_administrators_user_account}{Log in with the administrator’s user account}\grqq\ kann die Passwort-Validierung bei einem Login umgangen werden.

Vollziehen Sie das beschriebene Problem nach:
\begin{itemize}
    \item Was muss in der Login Maske angegeben werden, um als administrator eingeloggt zu werden?
    \item Wie funktioniert der Login Prozess im Backend?
    \item Überlegen Sie sich Funktionierende Abwandlungen des Befehls zum Login als Administrator.
    \item Was haben diese Gemeinsam?
\end{itemize}

\textbf{Schreiben Sie eine WAF-Regel, die es ermöglicht den Angriff zu verhindern, indem Anfragen die den Login Bypass enthalten, abgelehnt werden.
Bedenken Sie, dass nicht nur ein Weg zu einer Ausnutzung führen kann.}

\subsubsection{SQL-Injections allgemein an drei Bespielen}

Der Juice Shop bietet weitere Möglichkeiten für SQL injections.
Zwei Beispiele sind die beiden Challenges \glqq\href{https://pwning.owasp-juice.shop/companion-guide/latest/part2/injection.html#_exfiltrate_the_entire_db_schema_definition_via_sql_injection}{\underline{database schema extraction}}\grqq\ und \glqq\href{https://pwning.owasp-juice.shop/companion-guide/latest/part2/injection.html#_log_in_with_benders_user_account}{log in with Bender’s user account}\grqq.

\begin{itemize}
    \item Welche SQL-Keywords werden in den Angriffen genutzt?
    \item Welche Mutationen dieser Keywords können verwendet werden und der Angriff funktioniert immer noch?
    \item Welche Probleme können Auftreten wenn die SQL-Keywords zu streng blockiert werden?
\end{itemize}

\textbf{Schreiben sie eine Regel die beide Angriffe aus den Challenges anhand der SQL Keywords verhindern kann. }

\subsection{Cross Site Scripting (XSS)}

Cross Site Scripting (XSS) ist eine weiter Klasse an Angriffen die in der aktuellen Cybersicherheit eine große Rolle Spielen.
Eine XSS-Schwachstelle zeichnet sich dadurch aus, dass Nutzereingaben die HTML, CSS oder JavaScript enthalten ausgeführt oder als Teil der Webseite dargestellt werden.

\subsubsection{Reflected XSS}

Die Challenge \glqq\href{https://pwning.owasp-juice.shop/companion-guide/latest/part2/xss.html#_perform_a_persisted_xss_attack_bypassing_a_server_side_security_mechanism}{Perform a persisted XSS attack bypassing a server-side security mechanism}\grqq\ enthält eine persistente XSS Schwachstelle.
Das heißt ein in HTML eingebetteter JavaScript Code-Abschnitt kann auf dem Server gespeichert werden und wird bei Aufruf auf einem Beliebigen client ausgeführt.

\begin{itemize}
    \item Was macht eine XSS-Schwachstelle aus?
    \item Können andere Elemente als \textit{iframes} und dem \textit{javascript:alert} genutzt werden?
\end{itemize}\

\textbf{Schreiben Sie eine Regel die das Ausnutzen der Schwachstelle verhindert.
Erweitern sie die Regel um eine Rewrite direktive, die den schadhaften Abschnitt aus der Anfrage entfernt sie aber an den Server weitergibt.}

\subsubsection{XSS Protection HTTP Headers}

Neben dem erkennen von XSS durch einen WAF existieren den HTTP-Header \textit{X-XSS-Protection} der in Browsern eie XSS-Schutzfunktion aktiviert.

\textbf{Schreiben Sie eine Regel die in allen Antworten des JuiceShops diesen Header setzt.}

\pagebreak