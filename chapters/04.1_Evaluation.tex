Um einschätzen zu können wie die Lerneinheiten in der Lehre eingesetzt werden könne wird eine Evaluation durchgeführt.
Wie viel Zeit wird für die Durchführung der Übungen benötigt?
Welche Vorkenntnisse sind notwendig um die Übungen erfolgreich durchführen zu können?
Welches wissen erwerben die Lernenden in der Laborumgebung?

Bevor die Laborumgebung begonnen wird, soll eine Selbsteinschätzung durchgeführt werden in der die Kenntnisse zu bestimmten Technologien abgefragt werden, die in der Laborumgebung relevant sind.
Die Teilnehmer sollen auf einer Skala von 1 bis 6 ihre Fähigkeiten in den folgenden Technologien bewerten:

\begin{itemize}
    \item Linux
    \item Virtuelle Maschinen
    \item Containervirtualisierungsumgebung Docker
    \item Softwareentwicklung
    \item Reguläre Ausdrücke
    \item HTTP auf Protokollebene
    \item SQL-Syntax
    \item Sicherheitslücken in Webanwendung
    \item Web Application Firewall
\end{itemize}

Nach der Selbsteinschätzung werden die Lerneinheiten durchgeführt.
Die Probanden sollen die Lerneinheiten eigenständig bearbeiten damit das Umfeld möglich nah an dem geplanten Einsatzgebiet der Laborumgebung ist.



\subsection{Evaluation mit Probanden}

Die Evaluation 

% Zeit
% Scheitern 
% da war noch was drittes


