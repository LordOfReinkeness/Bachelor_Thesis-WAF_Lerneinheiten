Um einschätzen zu können, wie die Lerneinheiten in der Lehre eingesetzt werden könne wird eine Evaluation durchgeführt.
Die zentralen Fragen der Evaluation sind:
\begin{enumerate}
    \item Wie viel Zeit wird für die Durchführung der Übungen benötigt?
    \item Welche Vorkenntnisse sind notwendig, um die Übungen erfolgreich durchführen zu können?
    \item Welches Wissen erwerben die Lernenden in der Laborumgebung?
\end{enumerate}

Bevor die Laborumgebung begonnen wird, soll eine Selbsteinschätzung durchgeführt werden, in der die Kenntnisse zu bestimmten Technologien abgefragt werden, die in der Laborumgebung relevant sind.
Die Teilnehmer sollen auf einer Skala von 1 bis 6 ihre Fähigkeiten in den folgenden Technologien bewerten:

\begin{itemize}
    \item Linux
    \item Virtuelle Maschinen
    \item Containervirtualisierungsumgebung Docker
    \item Softwareentwicklung
    \item Reguläre Ausdrücke
    \item \ac{http} auf Protokollebene
    \item SQL-Syntax
    \item Sicherheitslücken in Webanwendung
    \item Web Application Firewall
\end{itemize}

Nach der Selbsteinschätzung werden die Lerneinheiten durchgeführt.
Die Probanden sollen die Lerneinheiten eigenständig bearbeiten damit das Umfeld möglich nah an dem geplanten Einsatzgebiet der Laborumgebung ist.
Nach der Durchführung werden die Probanden in einem direkten Gespräch befragt und die erzielten Lösungen Analysiert.

Aus den Gesammelten Informationen lässt dich abschätzen welches Vorwissen für die Laborumgebung notwendig ist ist ob die geplante Zeitvorgabe realistisch ist.

\subsection{Evaluation mit Probanden}

Die Evaluation erfolgt mit vier Probanden.
Alle Teilnehmer sind im Bereich der der IT-Sicherheit tätig oder in der Ausbildung zum Fachinformatiker.
Dadurch haben alle Probanden ein Verständnis für die Funktion von IT-Systemen, jedoch in in unterschiedlicher Tiefe.

Die Gruppe Lässt sich in zwei Teile aufteilen.
Diejenigen, die in der vorhergegangen Befragung angegeben haben keine oder geringe Kenntnisse in den Themengebieten \textit{Reguläre Ausdrücke} und \textit{\ac{http}} waren nicht in der Lage die Aufgaben erfolgreich abzuschließen.
Vorwissen in diesen beiden Themengebieten lässt sich also als Voraussetzung für die Durchführung betrachten.
Es lässt sich Vermuten, dass das Gleiche für Probanden ohne Kenntnisse mit SQL Probleme bei der Durchführung haben werde.
Bei alle Probanden war dieses wissen jedoch Verfügbar und diese These lässt sich nicht betätigen.
Es scheint also empfehlenswert, dass Probanden mit wissen über Reguläre Ausdrücke, der Funktion von \ac{http} und Kenntnisse der SQL-Syntax in die Lerneinheit einsteigen.

Die andere beiden Probanden waren in der Lage die Laborumgebung ohne Einschränkung durchzuführen.
Einer der Probanden, der im Bereich \ac{waf} tätig ist, war in der Lage die Laborumgebung innerhalb von drei Stunden durchzuführen.
Bei dem der letzte Proband, dessen Wissen dem eines Studenten am besten entspricht, hat die Laborumgebung in einem vollen Arbeitstag abgeschlossen.
Aus diesen Begrenzten Informationen lässt sich Vermuten, dass einen Durchführung der Lerneinheit im Rahmen einer Vorlesung mit eigenarbeitszeit möglich ist.

%ToDo: Setup