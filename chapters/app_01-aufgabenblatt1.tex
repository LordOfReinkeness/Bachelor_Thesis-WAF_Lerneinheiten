Mit dieser Laborumgebung soll die Funktion einer Web Application Firewall (WAF) in drei Lerneinheiten vermittelt werden.
In dieser Lerneinheit sollen Sie sich mit der Lernumgebung vertraut machen und eine erste Konfiguration der WAF vorm=nehmen.

\subsection{Vorbereitungen}

Um die Laborumgebung zu nutzen werden die Folgenden Anwendungen benötigt:
\begin{itemize}
    \item \href{https://www.virtualbox.org/}{VirtualBox}
    \item \href{https://code.visualstudio.com/download}{Visual Studio Code}\\
    Die nicht quelloffene Version von Microsoft ist notwendig, da ein benötigtes Plugin in \textit{Open-Source} Distributionen wie \textit{vscodium} nicht verfügbar sind.
    \item Einen Web-Browser
\end{itemize}

\subsubsection{Virtuelle Maschine starten}
Es sind Konfigurationen notwendig um 

\subsubsection{Websites aufrufen}
\subsubsection{Die WAF Konfigurationsdatei bearbeiten}

\subsection{Erste Konfiguration}

\subsubsection{}
\subsubsection{}



\pagebreak