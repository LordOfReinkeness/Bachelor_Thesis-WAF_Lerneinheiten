% \ac{nat} (Network Address Translation)
% - Layer 2 & 3 (IP & Port)
% - Internetzugang
% - Funktion: 
%     - Zuordnung von IP-Adressen: n private IPs -> 1 öffentlichen IP.
%     - Adressenspeicherung
%     - Einfachheit des Netzes
% - Sicherheit:
%     - grundlegend
%     - Verbirgt die interne Netzwerkstruktur
%     - keine Verschlüsselung
% 
% Reverse Proxy
% - Layer 5 (HTTP, HTTPS, FTP,\dots)
% - Webanwendungen, CDNs, Schutz interner Netzwerke
% - Funktion:
%     - Abfrageweiterleitung von Clients -> Server
%     - Lastausgleich
%     - Server-Anonymität
%     - Caching
% - Sicherheit:
%     - SSL-Terminierung
%     - Server-Identitäten verbergen
%     - Grundlegende Normalisierungs-Operationen
% 
% Vergleich:
% - \ac{nat}: Netzwerkebene <-> Reverse Proxy: Anwendungsebene 
% - Sicherheit: 
%     - \ac{nat}: quasi nicht
%     - Reverse Proxy: Verschlüsselung & Normalisierung
%     - Komplexität: Reverse Proxy -> Konfigurationsaufwand
% 
% WAF Nächste stufe
% Konzeptionell Reverse Proxy Modul
Die Funktion einer \ac{waf} baut auf der Fähigkeit aus Internet-Trafic weiterleiten und inspizieren zu können.
Die Technische Grundlage hierfür ist ein Reserve Proxy.
Im Folgenden wird beschrieben wie diese Technologie funktioniert, wie schon durch den Einsatz eines Proxies erste Sicherheitsaspekte zum Tragen kommen und wie sich ein Reverse Proxy von einfacheren Technologien wie \ac{nat} unterscheiden.

\paragraph{Network Acces Translation (NAT)}\ \\
\ac{nat} ist der Oberbegriff für Funktionen mit denen es möglich ist, die IP Adressen von Netzwerk-Traffic umzuschreiben.
Das heißt, Traffic kann über einen Relay-Punkt an unterschiedliche Clients oder Server weitergeleitet werden.
Es wird auf den Schichten zwei und drei des TCP/IP Schichtenmodells gearbeitet.
Ein Einsatzgebiet für \ac{nat} ist zum Beispiel die Adress-Umverteilung an einem Router.
Dadurch besteht die Möglichkeit an einer IP-Adresse mehrere Server zu betreiben.
Anfragen werden, je nachdem an welchem Port sie eintreffen, an eine Unterschiedliche IP-Adressen \textit{hinter} dem \ac{nat}-Gerät weitergeleitet.

Diese Technologie ist in seiner Umsetzung sehr niedrig Komplex, ermöglicht jedoch so gut wie keine Sicherheitsvorteile.
Der einzige kleine Vorteil kann die Verschleierung einer Internen Netzwerkstruktur sein.
Es ist nicht möglich tiefer gehende Filterung vorzunehmen, da auf einer ISO/OSI Schicht gearbeitet wird auf der die Entschlüsselung von Inhalten nicht möglich ist.
Protokolle die Verschlüsselung anbieten arbeiten auf Höheren Schichten.

\paragraph{Reverse Proxy}\ \\
Ein Reverse Proxy ist in seiner Funktion deutlich Komplexer als \ac{nat}.
Operiert wird auf Stufe vier des TCP/IP Schichtenmodells.
Ein Reverse Proxy fungiert, wie ein \ac{nat} fähiges Gerät, als Weiterleitungsstelle für Netzwerk-Verkehr.
Der Unterschied liegt jedoch auf der Protokoll-Ebene.
Ein Reverse Proxy ist in der Lage höher-Komplexe Protokolle wie HTTP, HTTPS oder FTP zu verstehen und die Weiterleitung aufgrund dem Inhalt der jeweiligen Pakete vorzunehmen.

Hieraus ergeben sich einige Sicherheitsaspekte die durch einen Reverse Proxy entstehen.
Da ein Reverse Proxy den Inhalt der HTTP Pakte inspiziert und weiterleitet kann eine Normalisierung der Anfragen erfolgen.
Dies kann eine Verteidigung gegen Path-Traversal Attacken sein, bei denen sich Zugang zu Inhalten verschafft werden kann die nicht Veröffentlicht werden sollten.

\pagebreak