\label{sec:introuduction}

% Raffinesse von Angriffen 
% Wettrüsten zwischen Angreifern und Verteidigern
% -> Sicherheitslücken/schlechte Programmierung          
% sicherheit
% Lösung waf
In einer Zeit, in der das Internet und die darauf basierenden Technologien in fast allen Bereichen der Wirtschaft und Gesellschaft zunehmend wichtiger sind, ist die Verteidigung dieser Technologien gegen Angreifer ein wichtiges und präsentes Thema.
Im Wettrüsten zwischen Verteidigern und Angreifern muss sich konstant auf neue Angriffstechniken eingestellt werden.
Als die ersten Server öffentlich im Internet verfügbar waren, war es üblich dies ohne jegliche Filterung zu tun.
Nach und nach mussten die Verteidigungen aufgerüstet werden.
Inzwischen ist die bloße Filterung auf Port-Ebene, um den Zugriff auf bestimmte Services zu beschränken, nicht mehr ausreichend.
Angreifer nutzen Schwachstellen in den öffentlich zugänglichen Anwendungen aus um ihre Ziele zu Erreichen.
Es ist notwendig den Inhalt der Netzkommunikation selber zu analysieren.

% WAF relevant weil:
%   markt wachstum
%   DSGVO implizit relevant

Deswegen gewinnt das Feld der \ac{waf} aktuell immer mehr an Relevanz.
Dem Markt wird in den nächsten fünf Jahren ein jährliches Wachstum um 19,9 \% auf 14,6 Mrd.\$ vorhergesagt \cite{WebApplicationFirewall}.
Auch Ausarbeitungen zum \textit{Stand der Technik} wie sie zum Beispiel im Deutschen \ac{itsig} und der \ac{dsgvo} gefordert werden, beschreiben eine \ac{waf} als notwendig zur Absicherung einer Webanwendung\cite[3.1.19 Schutz von Webanwendungen]{StandTechnik}.

% was soll vermitttelt werden?
%   was mach WAF
%   wie nutzt man WAF
%   wie greift man WAF an

Diese Bachelor Thesis hat die Erarbeitung von Lerneinheiten und einer Laborumgebung anhand derer das Thema \ac{waf} vermittelt werden kann als Ziel.
Es werden Design beziehungsweise Implementierung einer \ac{waf} betrachtet.
Auch werden Themen vermittelt, die für den Betrieb einer \ac{waf} in einem produktiven Umfeld relevant sind.
Dazu zählen zum Beispiel Deployment und Anpassung einer \ac{waf} an die zu schützende Webanwendung.

%% TODO

\pagebreak