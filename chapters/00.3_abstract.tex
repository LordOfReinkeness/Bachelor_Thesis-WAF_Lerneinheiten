\section{Abstract}
Diese Thesis befasst sich mit dem Entwurf und der Umsetzung von Lerneinheiten zum Thema Web Application Firewall (WAF), ein zunehmend relevantes Feld im Bereich der Netzwerksicherheit. 
Die Motivation für diese Arbeit liegt in der wachsenden Bedeutung der Sicherheit von Webanwendungen und der Notwendigkeit, fundiertes Wissen über WAFs in der Lehre zu vermitteln.

Zentrale Ziele der Arbeit umfassen die Entwicklung einer Laborumgebung, die es den Lernenden ermöglicht, das Konzept und die Funktionsweise von Web Application Firewalls praktisch zu erlernen und anzuwenden. 
In dieser Thesis liegt der Fokus auf dem Entwurf, der Auswahl geeigneter Tools und Anwendungen sowie der technischen Realisierung dieser Lernumgebung.

In der Evaluation wird die Wirksamkeit und Praktikabilität der entwickelten Lernumgebung untersucht. 
Es wird erläutert, wie die Umgebung mit einer Gruppe von Probanden getestet wurde und welche Methoden zur Bewertung der Lernerfolge herangezogen wurden. 

\pagebreak