Dieses Kapitel beinhaltet die Beschreibung der Laborumgebung deren Erstellung eines der Hauptziele dieser Thesis ist.
Es werden zuerst Abwägungen zu den Inhalten, die übermittelt werden sollen, angestellt.
Des Weiteren werden Produkte evaluiert die für die Realisierung der Laborumgebung genutzt werden können und deren Nutzung in der Laborumgebung beschrieben.
Es wird sowohl eine \ac{waf} als auch eine Anwendung benötigt, die zu schützende Schwachstellen aufweist.
Final werden die erstellten lerneinheiten beschrieben und Abwägungen angestellt wie die lerneinheiten die erarbeiteten Leerinhalte übermitteln können.

\subsection{Zu übermittelnde Inhalte}
\label{sec:inhalte}


%Wird eine \ac{waf} als Schutz einer Webanwendung eingesetzt unterscheidenden sich die Aufgaben die für den Betrieb ausgeführt werden müssen deutlich von den Inhalten, die 
Der Fokus der Lerneinheiten soll auf der Funktion einer \ac{waf} liegen.
Die Lerneinheiten sind nicht geeignet um das Aufgabenfeld von \ac{waf}-Consultants zu vermitteln.
Der Fokus liegt auf dem Erkennen, Verstehen und abwehren von Cyber-Angriffen.
Im Betrieb einer \ac{waf} wird das hierfür notwendige Regelwerk mit den Produkten vorkonfiguriert ausgeliefert.
Die Aufgaben in diesem Fall sind die Reaktion und Adaption der \ac{waf} auf entstehende Fehler um die Nutzung der zu schütztenden Webseite ohne Funktions-Einschränkungen zu ermöglichen.

In dieser Laborumgebung sollen diese vorkonfigurierten Regeln von den Lernenden erarbeitet werden.
Die Entstehende Konfiguration ist höchst spezialisiert und kann in einem realen Szenario nicht ansatzweise Sicherheitsvorteile erbringen.
Die bei einer Produktiven-\ac{waf} mitgelieferten Regelwerke werden von Mathematikern oder Theoretischen Informatikern erstellt und sind deutlich Allgemeingültiger als diejenigen die in dieser Laborumgebung erarbeitet werden.\\

Die Laborumgebung besteht aus drei, aufeinander aufbauenden Lerneinheiten.
In einem erste Schritt soll, nachdem sich mit dem Aufbau der Laborumgebung vertraut gemacht wird, die generelle Position einer \ac{waf} im Netzwerkverkehr beschrieben werden.
Es sollen unkompliziert zugängliche Inhalte des \ac{http}-Protokolls analysiert werden um den generellen Aufbau einer einer Regel und die schritte der Verarbeitung in einer \ac{waf} zu Verstehen.\\

Die Zweite Lerneinheit beschäftigt sich mit grundlegenden Angriffen die von einer \ac{waf} abgefangen werden können.
Hierfür werden einige grundlegende Schwachstellen herangezogen die sowohl in der Realität auftreten als auch ohne Vorkenntnisse mit Sicherheitslücken in Webanwendung verständlich sind.
Die benötigten Vorkenntnisse sollen nur im Bereich der Software-Entwicklung und der Erstellung von Webseiten sowie einem Grundverständnis des \ac{http}-Protokolls liegen.
Anhand dieser Kenntnisse und der Beschreibung von Schwachstellen in der zu schütztenden Webanwendung soll ich ein Verständnis der Angriffsvektoren erarbeitet werden und ein Regelwerk erstellt werden, das diese abdeckt und die Ausnutzung vereitelt.